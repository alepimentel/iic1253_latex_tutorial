%!TEX program = xelatex
\documentclass[10pt,handout]{beamer}

\usetheme{metropolis}
\usepackage{booktabs}
\usepackage[scale=2]{ccicons}

\usepackage[spanish]{babel}

\usepackage{amsmath,amsfonts,graphicx}
\usepackage{minted}
\usepackage{xcolor}
\usepackage{multicol}
\usepackage{hyperref}

\newcommand{\bs}{\textbackslash}


\title[\LaTeX\hspace{2em}]{\LaTeX\ for dummies}
\author[A. Pimentel]{Alejandro Pimentel}
\date{\today}
\institute{Pontificia Universidad Católica de Chile}

\begin{document}
\maketitle

\section{Introducción}

\begin{frame}{Introducción}
  \begin{center}
    {\Large ¿Qué es \LaTeX?} \\ \pause
    \vfill
    Es un lenguaje de demarcación\footnote{Markup Language.} (como HTML).
  \end{center}
\end{frame}

\begin{frame}{¿Por qué \LaTeX?}
	\begin{itemize}
    \item El estándar en la comunidad científica\pause
    \item Orientado a la escritura matemática\pause
    \item Reutilización de código\pause
    \item Separación de formato y contenido\pause
    \item Innumerables librerías para todo\pause
    \item Referencias y Bibliografía\pause
    \item Free \& Open Source\pause
    \item Poner imagenes no arruina los documentos
	\end{itemize}
\end{frame}

\begin{frame}{Compilador}
  \begin{block}{Distribuciones}
    \begin{itemize}
      \item \href{http://miktex.org}\textbf{MiKTeX} para Windows
      \item \href{https://www.tug.org/texlive/}\textbf{TeX Live} para Linux y S.O. sobre UNIX
      \item \href{https://tug.org/mactex/}\textbf{MacTeX} es una redistribución de Tex Live para Mac.
    \end{itemize}
  \end{block}
\end{frame}

\begin{frame}{Editores}
  \begin{block}{Editores}
    \begin{itemize}
      \item \textbf{Bloc de Notas} o cualquier editor de texto
      \item \textbf{\href{http://texstudio.sourceforge.net}{TexStudio}}
    \end{itemize}
  \end{block}
  \begin{exampleblock}{..o editores en línea!}
    \begin{itemize}
      \item \href{http://sharelatex.com}{\textbf{ShareLaTeX}}
      \item \href{http://overleaf.com}{\textbf{Overleaf}}
    \end{itemize}
  \end{exampleblock}
\end{frame}

\begin{frame}[fragile]{Hello World!}
  \begin{minted}[gobble=2]{latex}
    \documentclass{article}

    \begin{document}
        Hello World!
    \end{document}
  \end{minted}
\end{frame}

\section{Basics}

\begin{frame}{Componentes de un documento \LaTeX}
  \begin{enumerate}
    \item Document Class
    \item Preamble
    \item El documento
    \item Section, subsection y subsubsection
    \item Environments
    \item Comandos
  \end{enumerate}
\end{frame}

\begin{frame}[fragile]{Document Class}
  \begin{itemize}
    \item La primera línea de todos los documentos \LaTeX
    \item Especifica el tipo del documento
  \end{itemize}
  \begin{minted}[gobble=4]{latex}
    \documentclass{article}
  \end{minted}
  Algunos tipos de documento:
  \begin{tabular}{| l | c |}
    \hline
    \textbf{Tipo de documento} & \textbf{Descripción} \\ \hline
    article & Documentos cortos y artículos de journal\footnote{Y sus tareas.}. \\ \hline
    report & Documentos largos. \\ \hline
    book & Libros \\ \hline
    letter & Cartas \\ \hline
    beamer\footnote{Como esta presentación} & Presentaciones en Beamer \\ \hline
  \end{tabular}
\end{frame}

\begin{frame}[fragile]{Preamble}
  Lo que está entre \emph{documentclass} y el \emph{contenido del documento}\footnote{Ver slide \ref{doc}}.
  Aquí va:
  \begin{itemize}
    \item Paquetes que se deben importar
    \item Definición de comandos personales
    \item Configuraciones de paquetes importados
  \end{itemize}
  \begin{minted}[gobble=4]{latex}
    \usepackage{amsmath,amsfonts,graphicx}
    \usepackage{showexpl,listings}
    \usepackage[usenames,dvipsnames]{xcolor}
  \end{minted}
\end{frame}

\begin{frame}[fragile]{El documento (\bs begin\{document\}..)}
  \label{doc}
  Es el \emph{environment}\footnote{Ver slide \ref{env}} donde   va el contenido del documento.
  \begin{minted}[gobble=4]{latex}
    \begin{document}
      NO HAY PAN!
    \end{document}
  \end{minted}
\end{frame}

\begin{frame}[fragile]{Section, subsection y subsubsection}
  Generalmente organizamos nuestros documentos de manera jerárquica. En \LaTeX, existen los siguientes comandos para separar las distintas partes del documento:
  \begin{enumerate}
    \item \bs section
    \item \bs subsection
    \item \bs subsubsection
    \item \bs paragraph
    \item \bs subparagraph
  \end{enumerate}
  \begin{minted}[gobble=4]{latex}
    \section{Una seccion numerada}
      Lorem ipsum..

    \section*{Una seccion no numerada}
      Lorem ipsum..
  \end{minted}
\end{frame}

\begin{frame}[fragile]{Environments}
  \label{env}
  Un \emph{environment} define o modifica el formato de su contenido. Este código,
  \begin{minted}[gobble=4]{latex}
    \begin{center}
      Hola, estoy centrado!
    \end{center}
  \end{minted}
  produce lo siguiente:
  \begin{center}
    Hola, estoy centrado!
  \end{center}
\end{frame}

\begin{frame}[fragile]{Comandos}
  Los comandos sirven para escribir símbolos especiales\footnote{Como los que no están en el teclado} y para muchas otras cosas. Por ejemplo:
  \begin{minted}[gobble=4]{latex}
    \textbf{Estoy en negrita.} \\
    \Large{Soy muy grande!} \\
    $\hat{a} = \frac{\vec{a}}{|\vec{a}|}$
    \includegraphics[width=.4\linewidth]{img/appa.jpg}
  \end{minted}
  \begin{multicols}{2}
    \textbf{Estoy en negrita.} \\
    {\Large Soy muy grande!} \\
    $\hat{a} = \frac{\vec{a}}{|\vec{a}|}$ \\
    \begin{center}
    \includegraphics[width=.4\linewidth]{img/appa.jpg}
    \end{center}
  \end{multicols}
\end{frame}


\section{Latex en sus tareas}

\begin{frame}{Math mode}
  \LaTeX\ facilita la escritura de fórmulas matemáticas, por ejemplo:
  \begin{equation*}
    \frac{1}{\displaystyle 1+
      \frac{1}{\displaystyle 2+
      \frac{1}{\displaystyle 3+x}}} +
      \frac{1}{1+\frac{1}{2+\frac{1}{3+x}}}
    \end{equation*}
\end{frame}

\begin{frame}[fragile]{Math mode: Inline}
  La ecuación va en línea con el texto:
  \begin{minted}[gobble=4]{latex}
    Sea $\Sigma$ un conjunto de proposiciones en
    logica proposicional y sea $\phi$ otra
    proposicion.
  \end{minted}
  Sea $\Sigma$ un conjunto de proposiciones en lógica proposicional y sea $\phi$ otra proposición.
\end{frame}

\begin{frame}[fragile]{Math mode: Display}
  La ecuación va centrada en una linea aparte:
  \begin{minted}[gobble=4]{latex}
    Demuestre que:
    $$A \cup B = B \cup A$$
  \end{minted}
  Demuestre que:
  $$A \cup B = B \cup A$$
\end{frame}

\begin{frame}{Algunos símbolos}
  Todos los símbolos matemáticos están en \LaTeX, aunque generalmente hay que usar paquetes\footnote{Como amsmath, amsfonts o amssymb.}.
  \begin{multicols}{2}
    \begin{center}
      \begin{tabular}{| c | c |}
        \hline
        \textbf{Símbolo} & \textbf{Comando} \\ \hline
        $\alpha$ & \bs alpha \\ \hline
        $\beta$ & \bs beta \\ \hline
        $\neg$ & \bs neg \\ \hline
        $\vee$ & \bs vee \\ \hline
        $\wedge$ & \bs wedge \\ \hline
        $\rightarrow$ & \bs rightarrow \\ \hline
        $\leftarrow$ & \bs leftarrow \\ \hline
        $\leftrightarrow$ & \bs leftrightarrow \\ \hline
        $\Leftrightarrow$ & \bs Leftrightarrow \\ \hline
      \end{tabular}
      \begin{tabular}{| c | c |}
        \hline
        \textbf{Símbolo} & \textbf{Comando} \\ \hline
        $\forall$ & \bs forall \\ \hline
        $\exists$ & \bs exists \\ \hline
        $\in$ & \bs in \\ \hline
        $\not \in$ & \bs not \bs in \\ \hline
        $\leq$ & \bs leq \\ \hline
        $\geq$ & \bs geq \\ \hline
        $\cup$ & \bs cup \\ \hline
        $\cap$ & \bs cap \\ \hline
        $\subset$ & \bs subset \\ \hline
        $\subseteq$ & \bs subseteq \\ \hline
      \end{tabular}
    \end{center}
  \end{multicols}
\end{frame}

\begin{frame}{Más símbolos y comandos}
  \begin{multicols}{2}
      \begin{center}
        \begin{tabular}{| c | c |}
          \hline
          \multicolumn{2}{| c |}{De tamaño variable} \\ \hline \hline
          \textbf{Símbolo} & \textbf{Comando} \\ \hline
          $\sum$ & \bs sum \\ \hline
          $\prod$ & \bs prod \\ \hline
          $\int$ & \bs int \\ \hline
          $\oint$ & \bs oint \\ \hline
          $\bigcup$ & \bs bigcup \\ \hline
          $\bigcap$ & \bs bigcap \\ \hline
          $\bigvee$ & \bs bigvee \\ \hline
          $\bigwedge$ & \bs bigwedge \\ \hline
          $\biguplus$ & \bs biguplus \\ \hline
        \end{tabular}
        \begin{tabular}{| c | c |}
          \hline
          \multicolumn{2}{| c |}{Comandos} \\ \hline \hline
          \textbf{Acción} & \textbf{Comando} \\ \hline
          Negrita & \bs textbf \\ \hline
          Cursiva & \bs textit \\ \hline \hline
          \multicolumn{2}{| c |}{Font size} \\ \hline \hline
          \textbf{Tamaño} & \textbf{Comando} \\ \hline
          Tiny & \bs tiny \\ \hline
          Small & \bs small \\ \hline
          Large & \bs large \\ \hline
          Larger & \bs Large \\ \hline
          Huge & \bs huge \\ \hline
        \end{tabular}
      \end{center}
  \end{multicols}
\end{frame}

\begin{frame}[fragile]{Simbolos y comandos matemáticos en acción}
  \begin{enumerate}
    \item $a_{i+2} =  a_i + a_{i+1}$
      \pause
      \begin{minted}[gobble=8]{latex}
        $a_{i+2} =  a_i + a_{i+1}$
      \end{minted}
      \pause
    \item $a^2 + b^2 = c^2$
      \pause
      \begin{minted}[gobble=8]{latex}
        $a^2 + b^2 = c^2$
      \end{minted}
      \pause
    \item $\{\forall a \in A \mid a \text{ es primo}\}$
      \pause
      \begin{minted}[gobble=8]{latex}
        $\{\forall a \in A \mid a \text{es primo}\}$
      \end{minted}
  \end{enumerate}
\end{frame}

\begin{frame}[fragile]{Simbolos y comandos matemáticos en acción}
  \begin{enumerate}
    \setcounter{enumi}{3}
    \item $$\sum_{k=0}^{n} k = \frac{n*(n+1)}{2}$$
      \pause
      \begin{minted}[gobble=8]{latex}
        $$\sum_{k=0}^{n} k = \frac{n*(n+1)}{2}$$
      \end{minted}
      \pause
    \item $$f(n) = \begin{cases} 1 & \text{if } n = 0 \\ f(n-1) * n & \text{if } n > 0. \end{cases}$$
      \pause
      \begin{minted}[gobble=8]{latex}
        $$f(n) =
        \begin{cases}
          1 & \text{if } n = 0 \\
          f(n-1) * n & \text{if } n > 0.
        \end{cases} $$
      \end{minted}
  \end{enumerate}
\end{frame}


\section{LaTeX like a boss}

\begin{frame}[fragile]{BibTex}
  Sirve para manejar referencias. Supongamos que el archivo \emph{refs.bib}\footnote{La base de datos de referencias.} contiene lo siguiente:
  \begin{minted}[gobble=4]{latex}
    @article{jlreuttermagic2002,
      author  = {Juan L. Reutter},
      title   = {Building a winning Deck},
      journal = {MtG},
      year    = {2002}
    }
  \end{minted}
  Para citar esta publicación, hay que poner:
  \begin{minted}[gobble=4]{latex}
    \cite{jlreuttermagic2002}
  \end{minted}
\end{frame}

\begin{frame}[fragile]{..BibTex}
  Y al final del documento debe ir:
  \begin{minted}[gobble=4]{latex}
    \bibliographystyle{acm}
    \bibliography{refs}
  \end{minted}
\end{frame}

\begin{frame}[fragile]{Un buen código debe estar ordenado (?)}
  Un documento \LaTeX\ puede estar compuesto por varios archivos distintos.
  \begin{minted}[gobble=4]{latex}
    \begin{document}
      \begin{center}
        {\huge Tarea 1}
      \end{center}

      \begin{enumerate}
        \item Pregunta 1\\
          \input{p1.tex}
        \item Pregunta 2\\
          \input{p2.tex}
      \end{enumerate}
    \end{document}
  \end{minted}
\end{frame}


\section{Environments útiles! :D}

\begin{frame}[fragile]{Array}
  \begin{minted}[gobble=4]{latex}
    \begin{equation*}
      \begin{array}{lcccr}
        a, b, c & \in & A_1 & \subset & A\\
        b, d, e, f, g, h & \in & A_2 & \subset & A
      \end{array}
    \end{equation*}
  \end{minted}
  \begin{equation*}
    \begin{array}{lcccr}
      a, b, c & \in & A_1 & \subset & A\\
      b, d, e, f, g, h & \in & A_2 & \subset & A
    \end{array}
  \end{equation*}
\end{frame}

\begin{frame}[fragile]{Enumerate}
  \begin{minted}[gobble=4]{latex}
    \begin{enumerate}
      \item Soy un item numerado.
      \item Yo tambien!
    \end{enumerate}
  \end{minted}
  \begin{enumerate}
    \item Soy un item numerado.
    \item Yo tambien!
  \end{enumerate}
\end{frame}

\begin{frame}[fragile]{Itemize}
  \begin{minted}[gobble=4]{latex}
    \begin{itemize}
      \item Solo me dieron un punto ordinario! :(
      \item También quiero un numero.
    \end{itemize}
  \end{minted}
  \begin{itemize}
      \item Solo me dieron un punto ordinario! :(
      \item También quiero un numero.
    \end{itemize}
\end{frame}

\begin{frame}[fragile]{Tabular}
  \begin{minted}[gobble=4]{latex}
    \begin{tabular}{ c | c | c }
      x & o & x \\ \hline
      o & x & o \\ \hline
      o &   &
    \end{tabular}
  \end{minted}
  \begin{center}
    \begin{tabular}{ c | c | c }
      x & o & x \\ \hline
      o & x & o \\ \hline
      o &   &
    \end{tabular}
  \end{center}
\end{frame}


\section{Paquetes útiles}

\begin{frame}[fragile]{Paquetes útiles}
  \begin{tabular}{ | c | c | }
    \hline
    \textbf{Paquete} & \textbf{Qué hace?} \\ \hline
    amsmath & Facilita la escritura de fórmulas \\ \hline
    amssymb & Agrega la mayoría de los caracteres matemáticos \\ \hline
    babel & Cambia el idioma de entrada del documento \LaTeX \\ \hline
    float & Hace que las imágenes vayan donde uno las pone \\ \hline
    fullpage & Ajusa los margenes de todos los bordes a 1.5 cm \\ \hline
  \end{tabular}
\end{frame}


\section{Links útiles}

\begin{frame}{Y cómo hago un backslash? :(}
  \LARGE{\url{http://detexify.kirelabs.org}}
\end{frame}

\begin{frame}{Links útiles}
  \begin{enumerate}
    \item \url{https://google.com}
    \item \url{http://sharelatex.com/learn}
    \item \url{http://tex.stackexchange.com}
    \item \url{http://en.wikibooks.org/wiki/LaTeX}
  \end{enumerate}
\end{frame}

\section{Fin}

\end{document}
